\documentclass[a4paper,12pt]{article}
\usepackage[utf8]{inputenc}
\usepackage{graphicx}
\usepackage{graphics}
\usepackage{hyperref}
\usepackage{geometry}
% \usepackage{comment}
\graphicspath{ {build/} } 
\geometry{left=2cm, right=2cm, top=2cm, bottom=2cm}

\begin{document}

\begin{titlepage}
    \begin{center}
        \vspace*{3cm}
        
        {\Huge \textbf{Laboratorio 1 - Grupo 7}}\\[1cm]
        {\LARGE Panel solar automático:\\ [0.5cm]Acta de constitución de Proyecto}\\[2cm]
        
        \vfill
        
        {\Large \textbf{Integrantes }}\\[.5cm]
        \large
        \begin{tabular}{c c}
            Gartner, Francisco Nehuen & 69864/6 \\
            Marchesotti, Guido Daniel & 69923/9 \\
            Rosa, Fausto Pablo & 69843/1 \\
        \end{tabular}
        
        \vspace{1cm}
        
        \begin{figure}[b]
            \centering
            \includegraphics[width=1\linewidth]{LOGOSFI-UNLP-color-01.png}
        \end{figure}
        
        %%{\large \today}
    \end{center}
\end{titlepage}

%%\newpage
%%\tableofcontents
%%\newpage

El siguiente documento constituye el acta de planificación del proyecto "Panel solar automático", cuyo propósito consistiría en desarrollar un sistema automatizado de seguimiento solar. Esta iniciativa tendría como objetivo optimizar la captación de energía por parte de un panel solar fotovoltaico, mediante la modificación dinámica de su orientación utilizando un sistema de control de 2 ejes. Tal mecanismo permitiría maximizar la potencia instantánea, esperando superar el rendimiento de los paneles con orientación fija y mejorando así la eficiencia energética del sistema en su conjunto.\\

Adicionalmente, se planearía la implementación de un sistema de monitoreo que almacenaría y transmitiría los datos recolectados sobre la potencia generada. Estos datos serían enviados a través de una conexión inalámbrica hacia una aplicación móvil, en la cual el usuario podría consultar información en una interfaz gráfica intuitiva. Esta aplicación permitiría visualizar los registros históricos de las mediciones, facilitando el análisis y la trazabilidad del desempeño energético.\\

\begin{figure}[h]
    \centering
    \includegraphics[width=0.7\linewidth]{diagrama_proyecto.png}
    \caption{Figura ilustrativa}
    \label{fig:enter-label}
\end{figure}

El desarrollo técnico del proyecto abarcaría distintas áreas clave: la implementación de un mecanismo de rotación para el panel, la gestión energética del microcontrolador, el diseño de un circuito de medición de potencia, el desarrollo de la comunicación entre dispositivos, y la creación de una aplicación móvil que funcione como interfaz gráfica. Para el correcto funcionamiento del mecanismo de rotación, se programaría el sistema de control encargado de ejecutar los ajustes de orientación del panel. Se integraría una fuente de alimentación con batería capaz de garantizar la autosuficiencia energética del dispositivo, permitiendo que la energía excedente sea almacenada.\\

El sistema deberá ser lo suficientemente robusto como para mantenerse funcional durante períodos prolongados sin intervención externa. La interfaz gráfica sería desarrollada con una lógica de uso simple y accesible, permitiendo su uso por operarios sin conocimientos técnicos avanzados. A su vez, el prototipo serviría como base para eventuales escalados o mejoras futuras, estableciendo un diseño replicable y con potencial de crecimiento.\\

Se plantea que el proyecto sea realizado en un periodo estimado de cinco meses. El proyecto se irá desarrollando en etapas de duración predeterminada, siguiendo un cronograma tentativo inicial.\\

El alcance del proyecto comprendería el diseño, construcción y validación de un prototipo funcional, incluyendo tanto el desarrollo del hardware como del software. No se contemplaría la fabricación a escala ni la comercialización del sistema, ya que esta etapa estaría centrada en demostrar la viabilidad del concepto mediante una unidad operativa.\\

Al finalizar el desarrollo, se entregarían los siguientes productos: el prototipo ensamblado y en funcionamiento, el código fuente del microcontrolador, la aplicación móvil y una guía básica de uso.
Se considerará exitoso si el prototipo logra seguir la trayectoria solar de forma autónoma durante un período prolongado, logra transmitir los datos a la aplicación sin errores y puede operar de forma autosuficiente bajo condiciones de uso normal.\\

Finalmente, se estimaría un presupuesto de insumos aproximado de \$160.000, contemplando el costo de sensores, microcontroladores, módulos de comunicación, componentes electrónicos, materiales de construcción y demás insumos necesarios. Esta estimación incluiría un margen de flexibilidad para cubrir ajustes y eventuales imprevistos que pudieran surgir durante el desarrollo del proyecto.\\
\end{document}
